%\documentclass[aps,onecolumn,preprint,superscriptaddress,nofootinbib,floats]{revtex4}
%\usepackage{graphicx}

\documentclass[12pt]{article}

\usepackage{amsmath}
\usepackage[linesnumbered,ruled,noline,noend]{algorithm2e}

%\usepackage[linesnumbered,noline,noend]{algorithm2e}
%\usepackage{amsmath}

\SetNlSty{}{}{}

\let\oldnl\nl% Store \nl in \oldnl
\newcommand\nonl{%
  \renewcommand{\nl}{\let\nl\oldnl}}% Remove line number for one line



\def\beq{\begin{equation}}
\def\eeq{\end{equation}}


\usepackage{amsmath,amssymb,graphicx,multirow,xspace,slashed}
\usepackage[colorlinks=true,urlcolor=blue,anchorcolor=blue,citecolor=blue,filecolor=blue,linkcolor=blue,menucolor=blue,pagecolor=blue]{hyperref}


\usepackage{floatrow}
% Table float box with bottom caption, box width adjusted to content
\newfloatcommand{capbtabbox}{table}[][\FBwidth]


\usepackage[font=footnotesize,labelfont=bf]{caption}

\newcommand*\myat{{\fontfamily{ptm}\selectfont @}}

%\usepackage[notref]{showkeys}
\usepackage{lineno}

\allowdisplaybreaks
\bibliographystyle{JHEP}

\addtolength{\oddsidemargin}{-.4in}
\addtolength{\evensidemargin}{-.4in}
\addtolength{\textwidth}{0.8in}
\addtolength{\topmargin}{-.6in}
\addtolength{\textheight}{1in}
%\addtolength{\footskip}{0.3in}
\renewcommand{\baselinestretch}{1.2}

\long\def\symbolfootnote[#1]#2{\begingroup%
\def\thefootnote{\fnsymbol{footnote}}\footnote[#1]{#2}\endgroup}

\renewcommand{\textfraction}{0}
\renewcommand{\topfraction}{0.95}


\newcommand{\newc}{\newcommand}
\newc{\gsim}{\lower.7ex\hbox{$\;\stackrel{\textstyle>}{\sim}\;$}}
\newc{\lsim}{\lower.7ex\hbox{$\;\stackrel{\textstyle<}{\sim}\;$}}
\newc{\gev}{\,{\rm GeV}}
\newc{\mev}{\,{\rm MeV}}
\newc{\ev}{\,{\rm eV}}
\newc{\kev}{\,{\rm keV}}
\newc{\tev}{\,{\rm TeV}}

\newcommand{\ifb}{\,\mathrm{fb}^{-1}}
\newcommand{\ipb}{\,\mathrm{pb}^{-1}}
\renewcommand*\descriptionlabel[1]{\hspace\labelsep\normalfont #1}

\def\ln{\mathop{\rm ln}}
\def\tr{\mathop{\rm tr}}
\def\Tr{\mathop{\rm Tr}}
\def\Im{\mathop{\rm Im}}
\def\Re{\mathop{\rm Re}}
\def\bR{\mathop{\bf R}}
\def\bC{\mathop{\bf C}}
\def\lie{\mathop{\hbox{\it\$}}} %pound sterling
\newc{\mz}{M_Z}
\newc{\mpl}{M_*}
\newc{\mw}{m_{\rm weak}}
\newc{\nr}[1]{N^c_R{}_{#1}}

%\renewcommand{\phi}{\varphi}

%indices and other greek stuff
\renewcommand{\a}{\alpha}
\newcommand{\da}{{\dot \alpha}}
\renewcommand{\b}{\beta}
\newcommand{\db}{{\dot\beta}}
\newcommand{\g}{\gamma}
\newcommand{\dg}{{\dot\gamma}}
\renewcommand{\d}{\delta}
\newcommand{\dd}{{\dot\delta}}
\newcommand{\m}{\mu}
\newcommand{\n}{\nu}
\newcommand{\e}{\epsilon}
\newcommand{\s}{\sigma} 
\renewcommand{\r}{\rho}
\newcommand{\bs}{{\bar\sigma}}
\renewcommand{\l}{\lambda}
\renewcommand{\L}{\Lambda}
\renewcommand{\k}{\kappa}
\renewcommand{\th}{\theta}
\newcommand{\thb}{{\bar\theta}}
\newcommand{\D}{\Delta}
\newcommand{\B}{\bar B_\mu}
\newcommand{\cA}{c_{A_{u,d}}}
\newcommand{\cH}{c_{m_{u,d}}}
\renewcommand{\dag}{\dagger}
\newcommand{\bra}{\langle}
\newcommand{\ket}{\rangle}
\newcommand{\Q}{\bar Q}
\renewcommand{\O}{O}

\newcommand{\CM}{{\mathcal M}}

%%%%%%%%%%%%%%%%%%%%%%%% special abrev's %%%%%%%%%%%%%%%%%%%%%%%%%%%%%

\newcommand{\mhu}{{\hat m_{H_u}}}
\newcommand{\mhd}{{\hat m_{H_d}}}
\newcommand{\mhud}{{\hat m_{H_{u,d}}}}


%%%%%%%%%%%%%%%%%%%%%%% latex eqn abrev's %%%%%%%%%%%%%%%%%%%%%%%%%%%%

\def\beq{\begin{equation}}
\def\eeq{\end{equation}}
\newcommand{\bea}{\begin{eqnarray}\begin{aligned}}
\newcommand{\eea}{\end{aligned}\end{eqnarray}}
\def\bitem{\begin{itemize}}
\def\eitem{\end{itemize}}
%
%
%%%%%%%%%%%%%%%%%%%%%%% common abrev's %%%%%%%%%%%%%%%%%
%
%

\newc{\ie}{{\it i.e.}}          \newc{\etal}{{\it et al.}}
\newc{\eg}{{\it e.g.}}          \newc{\etc}{{\it etc.}}
\newc{\cf}{{\it c.f.}}
\newcommand{\kahler}{K\"{a}hler }

\newcommand{\lang}{\mathcal{L}}
\newcommand{\C}{\mathbb{C}}
\newcommand{\CO}{O}
\newcommand{\half}{\frac{1}{2}}


%\renewcommand{\epvar}{\varepsilon}
%\renewcommand{\phi}{\varphi}
\renewcommand{\topfraction}{0.85}
\renewcommand{\textfraction}{0.1}
\renewcommand{\floatpagefraction}{0.75}

%number equations by section
 %\numberwithin{equation}{section}

%toc depth
\setcounter{tocdepth}{2}

%Begin special definitions for Instructions file
\newcommand{\ttbs}{\char'134}%\backslash for \tt
\newcommand\fverb{\setbox\fverbbox=\hbox\bgroup\verb}
\newcommand\fverbdo{\egroup\medskip\noindent%
            \fbox{\unhbox\fverbbox}\ }
\newcommand\fverbit{\egroup\item[\fbox{\unhbox\fverbbox}]}
\newbox\fverbbox
\newcommand{\jhepname}{JHEP}
%end


\renewcommand{\arraystretch}{1.3}

%\usepackage[usenames,dvipsnames]{xcolor}


\usepackage{tikz}

\newcommand{\shrug}[1][]{%
\begin{tikzpicture}[baseline,x=0.8\ht\strutbox,y=0.8\ht\strutbox,line width=0.125ex,#1]
\def\arm{(-2.5,0.95) to (-2,0.95) (-1.9,1) to (-1.5,0) (-1.35,0) to (-0.8,0)};
\draw \arm;
\draw[xscale=-1] \arm;
\def\headpart{(0.6,0) arc[start angle=-40, end angle=40,x radius=0.6,y radius=0.8]};
\draw \headpart;
\draw[xscale=-1] \headpart;
\def\eye{(-0.075,0.15) .. controls (0.02,0) .. (0.075,-0.15)};
\draw[shift={(-0.3,0.8)}] \eye;
\draw[shift={(0,0.85)}] \eye;
% draw mouth
\draw (-0.1,0.2) to [out=15,in=-100] (0.4,0.95); 
\end{tikzpicture}}



\newcommand{\MYhref}[3][blue]{\href{#2}{\color{#1}{#3}}}%

\begin{document}
%
%\title{Toy Generative Model for Jets}
%\author{Kyle Cranmer, Sebastian Macaluso and Duccio Pappadopulo}
%\maketitle


\begin{center}

\vskip 1cm
{\Large \bf Kinematic considerations for a Toy Generative Model for Jets}





\vskip0.2cm{}

\vskip 1.0cm
{\large $\text{Kyle Cranmer}^1$, $\text{Sebastian Macaluso}^1$ and $\text{Duccio Pappadopulo}^2$}
\vskip 0.6cm
{\small \it 1 Center for Cosmology and Particle Physics $\&$ Center for Data Science, New York University, USA} \\
{\small \it 2 Bloomberg LP, New York, NY 10022, USA.}
\vskip 2.0cm

\end{center}




%\section{Introduction}

%Start with a simple introduction about the motivation eg. a standalone description of a generative model to aid in ML research that is: easy to describe, has a tractable likelihood, captures the essential ingredients of the Parton shower, has python implementation with few software dependencies.
%




In this notes, we study the kinematics requirements and implications for a Toy Generative Model of Jets. There is a companion note describing the model.
Let start by considering a a 2-body decay. In the parent rest frame, we have the parent momentum $p^\mu_p=p^\mu_{\text{L}}+p^\mu_{\text{R}}=(\sqrt{s}, 0, 0, 0)$. From requiring 4-momentum conservation, the children energies are given by
\bea
E_{\text{L}}=\frac{\sqrt{s}}{2}\bigg(1+\frac{m_{\text{L}}^2}{s}-\frac{m_{\text{R}}^2}{s} \bigg) \\
E_{\text{R}}=\frac{\sqrt{s}}{2}\bigg(1+\frac{m_{\text{R}}^2}{s}-\frac{m_{\text{L}}^2}{s} \bigg)
\eea

and the magnitude of their 3-momentum by
\bea\label{eq:Prestframe}
|\vec{p}| =\frac{\sqrt{s}}{2} \bar{\beta}=\frac{\sqrt{s}}{2} \sqrt{1-\frac{2 (m_{\text{L}}^2+m_{\text{R}}^2)}{s}+\frac{(m_{\text{L}}^2-m_{\text{R}}^2)^2}{s^2}}
\eea



%%------------------------------------------------------------------------------------
\subsection{Simplest case:  $m_{\text{L}}=m_{\text{R}}=m$}
In this section we will specialize to the case where $m_{\text{L}}=m_{\text{R}}=m$ (we will comment on the general case in the next subsection).
\bea
E_{\text{L}}=E_{\text{R}}=\frac{\sqrt{s}}{2} = E \\
|\vec{p}| =E \sqrt{1-\frac{m^2}{E^2}}
\eea


If we  apply a boost to the lab frame, with factor $\beta \hat{n}$, we obtain for the children 3-momentum
\bea\label{eq:pChildLab}
\vec{p}^{\it{\,l}}_{\text{L}} = - \gamma E \beta \hat{n} \,-\,  E \sqrt{1-\frac{m^2}{E^2}} \,\,\,[ \hat{r} + (\gamma -1) (\hat{r} \cdot \hat{n}) \hat{n} ] \\
\vec{p}^{\it{\,l}}_{\text{R}}= - \gamma E \beta \hat{n} \,+\,  E \sqrt{1-\frac{m^2}{E^2}} \,\,\,[ \hat{r} + (\gamma -1) (\hat{r} \cdot \hat{n}) \hat{n} ]
\eea
with $\hat{r}$ a unit vector that specifies the direction of the children momentum in the parent rest frame. For the parent we have 
\bea\label{eq:pParentLab}
\vec{p}^{\it{\,l}}_p= - 2 \gamma E \beta \hat{n} 
\eea

From (\ref{eq:pParentLab}), we can rewrite (\ref{eq:pChildLab}) as:
\bea\label{eq:pChildLab2}
\vec{p}^{\it{\,l}}_{\text{L}}= \frac{1}{2} \vec{p}^{\it{\,l}}_p - \vec{\Delta}  \\
\vec{p}^{\it{\,l}}_{\text{R}}= \frac{1}{2} \vec{p}^{\it{\,l}}_p + \vec{\Delta}
\eea
with
\bea\label{eq:delta}
 \vec{\Delta}   = E \sqrt{1-\frac{m^2}{E^2}} \,\,\,[ \hat{r} + (\gamma -1) (\hat{r} \cdot \hat{n}) \hat{n} ] 
\eea

Let's study the dependence of $|\vec{\Delta}|=\Delta(E,m,\hat{r},\gamma,\hat{n})$. We will work in the (y,z) plane.

\begin{itemize}
\item $\gamma=\frac{E_p}{m_p}$ or $\gamma \beta = |\vec{p}_p|/m_p$
\item If we draw an angle $\phi \in \{-\pi,\pi\}$ from a uniform distribution, we get in the $(\hat{y},\hat{z})$ plane,  $\hat{r}= (\sin{\phi}, \cos{\phi})$.

\item At each step, we draw $\phi$ and $m$, and boost the children to the lab frame. Next, we promote each child to a parent and repeat the process. Thus, the unit vector $\hat{n}$ is fixed by the previous step and given by $\hat{n}=(\sin{\theta_\text{p}}, \cos{\theta_\text{p}})$ with $\theta_\text{p}=\tan^{-1}\bigg({\frac{(p_{\text{p}})_y}{(p_{\text{p}})_z}}\bigg)$ the parent angle with respect to the $\hat{z}$ axis in the lab frame. 

\item At each step, $E_{\text{L}}=E_{\text{R}}=E =\frac{\sqrt{s}}{2}= \frac{m_{\text{p}}}{2}$ is fixed by the parent mass.

\end{itemize}

As a result, we obtain $|\Delta|=\Delta(m,\phi)$ given by
\bea
 \Delta_y   &= \frac{m_\text{p}}{2} \sqrt{1- 4\frac{m^2}{m_\text{p}^2}}  \bigg[ \sin{\phi} + \bigg(\frac{m_\text{p}}{ 2 m}-1\bigg) \cos{(\phi-\theta_\text{p})} \sin{\theta_\text{p}} \bigg]\\ 
  \Delta_z   &= \frac{m_\text{p}}{2} \sqrt{1- 4\frac{m^2}{m_\text{p}^2}}   \bigg[\cos{\phi} +  \bigg(\frac{m_\text{p}}{2 m}-1\bigg) \cos{(\phi-\theta_\text{p})} \cos{\theta_\text{p}} \bigg]
\eea


%%---------------------------------------------------------------------------------------
\subsection{General case, with  $m_{\text{L}}\neq m_{\text{R}}$}

In this case, (\ref{eq:pChildLab2}) becomes
\bea\label{eq:pChildLab2Gen}
\vec{p}^{\,\it{l}}_{\text{L}}= \frac{E_{\text{L}}}{E_\text{p}} \vec{p}^{\,\it{l}}_\text{p} - |\vec{p}| \,\,\,[ \hat{r} + (\gamma -1) (\hat{r} \cdot \hat{n}) \hat{n} ]  \\
\vec{p}^{\,\it{l}}_{\text{R}}=  \frac{E_{\text{R}}}{E_\text{p}} \vec{p}^{\,\it{l}}_\text{p} +  |\vec{p}| \,\,\,[ \hat{r} + (\gamma -1) (\hat{r} \cdot \hat{n}) \hat{n} ] 
\eea
where $|\vec{p}|$ is as in (\ref{eq:Prestframe}):
\bea
|\vec{p}| =\frac{\sqrt{s}}{2} \bar{\beta}=\frac{\sqrt{s}}{2} \sqrt{1-\frac{2 (m_{\text{L}}^2+m_{\text{R}}^2)}{s}+\frac{(m_{\text{L}}^2-m_{\text{R}}^2)^2}{s^2}}\nonumber
\eea

\end{document}
